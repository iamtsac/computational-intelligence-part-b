\documentclass[12pt,a4paper]{article}

\usepackage[utf8]{inputenc}
\usepackage[greek,english]{babel}
\usepackage{float}
\usepackage[export]{adjustbox}
\usepackage{sans}
\usepackage{kerkis} 
%\usepackage{sans}
%\usepackage[LGRgreek]{mathastext}   
\usepackage{graphicx}
%\usepackage{enumerate}
\usepackage{enumitem}  
\usepackage{amsmath}
\usepackage{hyperref,xcolor} 
\usepackage{subcaption}

\hypersetup{
    colorlinks,
    linkcolor={black!50!black},
    citecolor={blue!50!black}, urlcolor={cyan!80!black},
}

\newcommand{\en}{\selectlanguage{english}} 
\newcommand{\tl}{\textlatin} 
\newcommand{\gr}{\selectlanguage{greek}}   
\newcommand{\code}[1]{\texttt{#1}}         
%\newcommand{\tsuper}{\textsuperscript} \newcommand{\tsub}{\textsubscript}

 \renewcommand{\thesection}{\arabic{section}.} 
% \renewcommand{\thesubsection}{\arabic{subsection}.}


\gr \title{{\bf \includegraphics[scale=1.0]{images/up_landscape.jpeg} \\ ΤΜΗΜΑ ΜΗΧΑΝΙΚΩΝ ΗΛΕΚΤΡΟΝΙΚΩΝ ΥΠΟΛΟΓΙΣΤΩΝ ΚΑΙ ΠΛΗΡΟΦΟΡΙΚΗΣ  \\ \vspace{3cm}Αναφορά Εργαστηριακής Άσκησης Μέρος Β' \\ Υπολογιστική Νοημοσύνη}}
\author{Κωνσταντίνος Τσάκωνας \\ Α.Μ.: 1059666}
\date{Ακαδημαϊκό έτος 2020-21\\ Εαρινό Εξάμηνο}

\begin{document}

    \gr \maketitle \newpage

    \tableofcontents  \newpage

    \section*{\tl{Repository} \gr Κώδικα}
        Για την ανάπτυξη της άσκσης χρησιμοποιήθηκε η γλώσσα \tl{Python} με τις βιβλιοθήκες \tl{Tensorflow, numpy, pandas, deap} και \tl{matplotlib}. Παρακάτω υπάρχει το \tl{repository} του κώδικα στο \tl{github} \\
        \underline{\tl{\textbf{\href{https://github.com/iamtsac/computational-intelligence-part-b}{github link}}}}

        \section*{Σχεδιασμός ΓΑ}
            \begin{enumerate}
                \item Τα άτομα του αρχικού πληθυσμού θα αναπαραστηθούν ως
                    δυαδικές συμβολοσειρές. Ο λόγος που θα ακολουθηθεί αυτή η
                    κωδικοποίηση προέρχεται από το σκεπτικό ότι, αυτό που
                    θέλουμε να κάνουμε είναι να μειώσουμε το είσοδους από τα 784
                    \tl{pixels}, δηλαδή να μηδενισούμε πολλες από αυτές.
                    Δημιουργώντας λοιπόν άτομα του πληθυσμού ως πίνακες 784
                    \tl{pixels} που περιέχουν δυαδικά στοιχεία, μπορούμε με ένα
                    πολλαπλασιασμό \tl{element-wise} να κρατήσουμε τις εισόδους που
                    θέλουμε.
                \item Ο αρχικός πληθυσμός θα είναι \tl{N} τυχαίοι πίνακες
                    28\tl{x}28 και οι τιμές που θα περιέχουν θα είναι 0 και 1.
                \item Η συνάρτηση καταλληλότητας που επιλέχθηκε είναι οι εξής:
                    \begin{itemize}
                        
                        \item κάθε φορά που θα γίνεται έλεγχος στο νευρωνικό
                            σύμφωνα με τις εισόδους που προκύπτουν από κάθε
                            άτομο του πληθυσμού θα κάνουμε ταξινόμιση των πρώτων
                            $10.000$ εικόνων και θα συγκρίνουμε την ταξινόμιση
                            αυτή με βάση τα \tl{labels} για να δούμε πόσο
                            ακριβής είναι. Το αποτελέσμα που θα προκύπτει από
                            αυτο θα είναι μια τιμή μεταξύ του διαστήματος
                            $[0,1]$
                            και στόχος του γενετικού αλγοριθμού θα είναι να
                            πλησιάσει όσο το δυνατό πιό κοντά στο $1$.
                        \item Επίσης θα επιβάλεται μία ποινή σε άτομα του
                            πληθυσμού που έχουν μεγάλο αριθμό εισόδων,
                            συγκεκριμένα άτομα που έχουν περισσότερες από $392$
                            εισόδους θα αφαιρείται μία τιμή η οποία θα είναι
                            ανάλογη με το ποσοστό που δεν ταξινομήθηκε σώστα επί
                            το πλήθος των παραπάνω εισόδων που έχει σε σχέση με
                            αυτές που έχουμε θέσει ως επιθυμητό άνω όριο.

                    \end{itemize}
                    Ο λόγος που επιλέχθηκε η μεγιστοποίηση του ποσοστού
                    ταξινόμισης έχει να κάνει με το γεγονός ότι κατά την δημιουργία και την
                    εκπαίδευση του μοντέλου στην προηγούμενη εργαστηριακή άσκηση
                    σε πάρα πόλυ μικρές τιμές του \tl{loss} η ταξινόμιση δεν
                    ήταν πάντα σωστή και δεν συμβάδιζε με το \tl{accuracy} του
                    μοντέλου. Ακόμα ο λόγος που εφαρμόζουμε ποίνη σε άτομα που
                    έχουν περισσότερες από $392$ εισόδους είναι γιατί θα έχουμε
                    μία μείωσει $50\%$ αλλά για λιγότερες εισόδους από αυτές θα
                    χάνουμε χαρακτηριστικά που χρειαζόμαστε αφού το σχήμα των
                    ψηφιών στις είκονες συνεχώς μεταβάλεται, με αποτέλεσμα να
                    μην έχουμε καλή ταξινόμιση.
            \end{enumerate}
\end{document}
