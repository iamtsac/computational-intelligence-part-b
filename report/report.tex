\documentclass[12pt,a4paper]{article}

\usepackage[utf8]{inputenc}
\usepackage[greek,english]{babel}
\usepackage{float}
\usepackage[export]{adjustbox}
\usepackage{sans}
\usepackage{kerkis} 
%\usepackage{sans}
%\usepackage[LGRgreek]{mathastext}   
\usepackage{graphicx}
\usepackage{enumerate}
%\usepackage{enumitem}  
\usepackage{amsmath}
\usepackage{hyperref,xcolor} 
\usepackage{subcaption}

\hypersetup{
    colorlinks,
    linkcolor={black!50!black},
    citecolor={blue!50!black}, urlcolor={cyan!80!black},
}

\newcommand{\en}{\selectlanguage{english}} 
\newcommand{\tl}{\textlatin} 
\newcommand{\gr}{\selectlanguage{greek}}   
\newcommand{\code}[1]{\texttt{#1}}         
%\newcommand{\tsuper}{\textsuperscript} \newcommand{\tsub}{\textsubscript}

 \renewcommand{\thesection}{\arabic{section}.} 
% \renewcommand{\thesubsection}{\arabic{subsection}.}


\gr \title{{\bf \includegraphics[scale=1.0]{images/up_landscape.jpeg} \\ ΤΜΗΜΑ ΜΗΧΑΝΙΚΩΝ ΗΛΕΚΤΡΟΝΙΚΩΝ ΥΠΟΛΟΓΙΣΤΩΝ ΚΑΙ ΠΛΗΡΟΦΟΡΙΚΗΣ  \\ \vspace{3cm}Αναφορά Εργαστηριακής Άσκησης Μέρος Α' \\ Υπολογιστική Νοημοσύνη}}
\author{Κωνσταντίνος Τσάκωνας \\ Α.Μ.: 1059666}
\date{Ακαδημαϊκό έτος 2020-21\\ Εαρινό Εξάμηνο}

\begin{document}

    \gr \maketitle \newpage

    \tableofcontents  \newpage

    \section*{\tl{Repository} \gr Κώδικα}
        Για την ανάπτυξη της άσκσης χρησιμοποιήθηκε η γλώσσα \tl{Python} με τις βιβλιοθήκες \tl{Tensorflow, numpy, scikit-learn, pandas} και \tl{matplotlib}. Παρακάτω υπάρχει το \tl{repository} του κώδικα στο \tl{github} και το \tl{link} ενός  \tl{shared} φακέλου στο \tl{google drive} σε περίπτωση που υπάρχει κάποιο πρόβλημα με το πρώτο. \\
        \underline{\tl{\textbf{\url{https://github.com/iamtsac/computational-intelligence-part-a}}}}a

\end{document}
